\documentclass[11pt]{article}
\usepackage{hyperref,amsmath,bm,apacite}

\begin{document}

{\large
\begin{center}
\noindent {\em Title:} Score-based tests of measurment invariance: Use
in practice\\ \ \\

\noindent {\em Authors:} Ting Wang, Edgar Merkle, Achim Zeileis
\end{center}}

\pagestyle{empty}

% Paragraph 1: Background
%% General statement about measurement invariance

%% Summary of Merkle + Zeileis 2013

%% The tests have had only limited application to measurement
%% invariance, and their properties as applied to measurement
%% invariance are not fully known.

As the Research Topic indicates, measurement invariance is a critical issue in
psychometric modeling. Recently, Merkle and Zeileis \citeyear{MerZei13} studied
a novel family of score-based tests for measurement invariance in factor
analysis models. This family can be used to test for measurement invariance
w.r.t.\ a continuous auxiliary variable without pre-specification of subgroups.
Moreover, the family of tests can be employed to identify subgroups of
observations and model parameters that violate measurement invariance. Merkle,
Fan, and Zeileis \citeyear{MerFanZei} extended this framework to testing
measurement invariance w.r.t.\ an ordinal auxiliary variable, yielding test
statistics that are sensitive to violations that are monotonically related to
the auxiliary variable (and much less sensitive to non-monotonic violations).

% Paragraph 2: Tutorial in lavaan
The papers cited in the above paragraph have been relatively technical and
slanted towards psychometricians. In the proposed paper for the Research Topic,
we plan to focus on the test statistics' use in practice. We will first
provide a minimally-technical overview of the tests, along with a tutorial on
carrying out the tests using the {R} packages \emph{lavaan} (for model estimation) and
\emph{strucchange} (for measurement invariance testing). These packages currently have
all the functionality necessary for carrying out the tests on arbitrary
structural equation models, so the tutorial will be especially useful for
researchers who wish to apply the tests to their own data.

%% Ting: I think the following details are too specific for the abstract.
% To carry out these tests, a confirmatory factor analysis model is fit to the data first. Then R package lavaan could
% provide casewise derivatives and observed information matrix, and then they are used to calculate the cumulative
% score process. In the end, by using different aggregation function of cumulative
% score process, we obtain three tests' statistics and \emph{p}-values, which could indicate whether or not measurement invariance is 
% violated.

% Paragraph 3: Simulations to examine issues that are of importance in
% practice.
Following the tutorial, we plan to conduct and report on some novel simulations
that are relevant to the tests' use in practice. In Simulation~1, we will study
the extent to which the proposed tests attribute measurement invariance
violations to the correct model parameters. In Simulation~2, we will study the
extent to which the tests are robust to model misspecification. Taken together,
the tutorial and simulations will provide researchers with both the means to
carry out the tests and the knowledge of when the tests are likely to be useful
in practice. 
%% Ting: For this proposed abstract, I think we do not need to discuss
%% any results.
% The resutls indicate that the tests
% could correctly locate the violation parameter, as long as the violation parameter is included
% in the model. If not, the tests show violation in more than one parameter associated with the
% particular manifest variable which is unduely modeled.

\bibliographystyle{apacite}
\bibliography{refs}

\end{document}
